polr {MASS}	R Documentation
Ordered Logistic or Probit Regression

Description

Fits a logistic or probit regression model to an ordered factor response. The default logistic case is proportional odds logistic regression, after which the function is named.

Usage

polr(formula, data, weights, start, ..., subset, na.action,
     contrasts = NULL, Hess = FALSE, model = TRUE,
     method = c("logistic", "probit", "cloglog", "cauchit"))
Arguments

formula	
a formula expression as for regression models, of the form response ~ predictors. The response should be a factor (preferably an ordered factor), which will be interpreted as an ordinal response, with levels ordered as in the factor. The model must have an intercept: attempts to remove one will lead to a warning and be ignored. An offset may be used. See the documentation of formula for other details.

data	
an optional data frame in which to interpret the variables occurring in formula.

weights	
optional case weights in fitting. Default to 1.

start	
initial values for the parameters. This is in the format c(coefficients, zeta): see the Values section.

...	
additional arguments to be passed to optim, most often a control argument.

subset	
expression saying which subset of the rows of the data should be used in the fit. All observations are included by default.

na.action	
a function to filter missing data.

contrasts	
a list of contrasts to be used for some or all of the factors appearing as variables in the model formula.

Hess	
logical for whether the Hessian (the observed information matrix) should be returned. Use this if you intend to call summary or vcov on the fit.

model	
logical for whether the model matrix should be returned.

method	
logistic or probit or complementary log-log or cauchit (corresponding to a Cauchy latent variable).

Details

This model is what Agresti (2002) calls a cumulative link model. The basic interpretation is as a coarsened version of a latent variable Y_i which has a logistic or normal or extreme-value or Cauchy distribution with scale parameter one and a linear model for the mean. The ordered factor which is observed is which bin Y_i falls into with breakpoints

zeta_0 = -Inf < zeta_1 < … < zeta_K = Inf

This leads to the model

logit P(Y <= k | x) = zeta_k - eta

with logit replaced by probit for a normal latent variable, and eta being the linear predictor, a linear function of the explanatory variables (with no intercept). Note that it is quite common for other software to use the opposite sign for eta (and hence the coefficients beta).

In the logistic case, the left-hand side of the last display is the log odds of category k or less, and since these are log odds which differ only by a constant for different k, the odds are proportional. Hence the term proportional odds logistic regression.

In the complementary log-log case, we have a proportional hazards model for grouped survival times.

There are methods for the standard model-fitting functions, including predict, summary, vcov, anova, model.frame and an extractAIC method for use with stepAIC (and step. There are also profile and confint methods.

Value

A object of class "polr". This has components

coefficients	
the coefficients of the linear predictor, which has no intercept.

zeta	
the intercepts for the class boundaries.

deviance	
the residual deviance.

fitted.values	
a matrix, with a column for each level of the response.

lev	
the names of the response levels.

terms	
the terms structure describing the model.

df.residual	
the number of residual degrees of freedoms, calculated using the weights.

edf	
the (effective) number of degrees of freedom used by the model

n, nobs	
the (effective) number of observations, calculated using the weights. (nobs is for use by stepAIC.

call	
the matched call.

method	
the matched method used.

convergence	
the convergence code returned by optim.

niter	
the number of function and gradient evaluations used by optim.

lp	
the linear predictor (including any offset).

Hessian	
(if Hess is true). Note that this is a numerical approximation derived from the optimization proces.

model	
(if model is true).

Note

The vcov method uses the approximate Hessian: for reliable results the model matrix should be sensibly scaled with all columns having range the order of one.

References

Agresti, A. (2002) Categorical Data. Second edition. Wiley.

Venables, W. N. and Ripley, B. D. (2002) Modern Applied Statistics with S. Fourth edition. Springer.

See Also

optim, glm, multinom.

Examples

options(contrasts = c("contr.treatment", "contr.poly"))
house.plr <- polr(Sat ~ Infl + Type + Cont, weights = Freq, data = housing)
house.plr
summary(house.plr, digits = 3)
## slightly worse fit from
summary(update(house.plr, method = "probit", Hess = TRUE), digits = 3)
## although it is not really appropriate, can fit
summary(update(house.plr, method = "cloglog", Hess = TRUE), digits = 3)

predict(house.plr, housing, type = "p")
addterm(house.plr, ~.^2, test = "Chisq")
house.plr2 <- stepAIC(house.plr, ~.^2)
house.plr2$anova
anova(house.plr, house.plr2)

house.plr <- update(house.plr, Hess=TRUE)
pr <- profile(house.plr)
confint(pr)
plot(pr)
pairs(pr)
