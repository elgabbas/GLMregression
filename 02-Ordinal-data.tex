\documentclass[00-GLMregslides.tex]{subfiles}
\begin{document}

%=========================================%
\begin{frame}
	\large
\textbf{Ordered Logistic Regression}\\
Examples of ordered logistic regression
\begin{itemize}
\item A marketing research firm wants to investigate what factors influence the size of soda (small, medium, large or extra large) that 
people order at a fast-food chain. 
\item These factors may include what type of sandwich is ordered (burger or chicken), whether or not fries are also ordered, and age of the consumer. 
\item While the outcome variable, size of soda, is obviously ordered, the difference between the various sizes is not consistent. 
\item The difference between small and medium is 10 ounces, between medium and large 8, and between large and extra large 12.
\end{itemize}
\end{frame}
%=========================================%
\begin{frame}
	\large
\textbf{Ordered Logistic Regression}\\
Examples of ordered logistic regression
\begin{itemize}
\item A researcher is interested in what factors influence medaling in Olympic swimming. 
\item Relevant predictors include at training hours, diet, age, and popularity of swimming in the athlete's home country. 
\item The researcher believes that the distance between gold and silver is larger than the distance between silver and bronze.
\end{itemize}
\end{frame}
%=========================================%
\begin{frame}
	\large
\textbf{Ordered Logistic Regression}

\begin{itemize}
\item 
A study looks at factors that influence the decision of whether to apply to graduate school. 
\item College juniors are asked if they are unlikely, somewhat likely, or very likely to apply to graduate school. 
\item Hence, our outcome variable has three categories. Data on parental educational status, whether the undergraduate institution is public or private, 
and current GPA is also collected. 
\item The researchers have reason to believe that the "distances" between these three points are not equal. 
\item For example, the "distance" between \textit{"unlikely"} and \textit{"somewhat likely"} may be shorter than the distance between \textit{"somewhat likely"} and \textit{"very likely"}.
\end{itemize}
\end{frame}
%=========================================%
\begin{frame}
\frametitle{Ordered Logistic Regression}
\large
Data Set - Graduate School Entry (ologit.csv) \smallskip
\begin{itemize}
\item 
This hypothetical data set has a three level variable called \textbf{\textit{apply}}, with levels \textit{"unlikely"}, \textit{"somewhat likely"}, and \textit{"very likely"}, coded 1, 2, and 3, respectively, 
that we will use as our outcome variable. 
% We also have three variables that we will use as predictors: pared, which is a 0/1 variable indicating whether at least 
% one parent has a graduate degree; public, which is a 0/1 variable where 1 indicates that the undergraduate institution is public and 0 private, and gpa, which is 
% the student's grade point average. 
% Let's start with the descriptive statistics of these variables.
\end{itemize}
\end{frame}
%=========================================%
\begin{frame}
\frametitle{Ordered Logistic Regression}
\large
Predictors:
\begin{description}
\item[pared], which is a 0/1 variable indicating whether at least 
one parent has a graduate degree; 
\item[public], which is a 0/1 variable where 1 indicates that the undergraduate institution is public and 0 private, 
\item[gpa], which is 
the student's grade point average. 
% Let's start with the descriptive statistics of these variables.
\end{description}
\end{frame}
%=========================================%
\begin{frame}[fragile]
\frametitle{Ordered Logistic Regression}
\large
\begin{verbatim}
            apply pared public  gpa
1     very likely     0      0 3.26
2 somewhat likely     1      0 3.21
3        unlikely     1      1 3.94
4 somewhat likely     0      0 2.81
5 somewhat likely     0      0 2.53
6        unlikely     0      1 2.59
\end{verbatim}
\end{frame}

\end{document}
