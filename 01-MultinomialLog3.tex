

%================================================================================================%
\documentclass[00-GLMregslides.tex]{subfiles}

\begin{document}

	

%================================================================================================%

\begin{frame}[fragile]

\frametitle{Multinomial Logistic Regression with \texttt{R}}
\Large
\vspace{-1cm}
\textbf{Odds Ratios}
\begin{itemize}
\item The ratio of the probability of choosing one outcome category over the probability of choosing the baseline category is often referred as 
\textbf{\textit{relative risk}} 
\item It is also sometimes referred as \textbf{\textit{odds}}.
% as we have just used to described the regression parameters above previously). 

\end{itemize}
\end{frame}
%================================================================================================%
\begin{frame}[fragile]
\frametitle{Multinomial Logistic Regression with \texttt{R}}	\Large
	\begin{itemize}
\item  The relative risk is the (right-hand side) linear equation exponentiated, leading to the fact that the 
exponentiated regression coefficients are relative risk ratios for a unit change in the predictor variable. 
\smallskip
\item  We can exponentiate the coefficients from our model to see these odds ratios (next slide). 
\end{itemize}
\end{frame}
%================================================================================================%
\begin{frame}[fragile]

\frametitle{Multinomial Logistic Regression with \texttt{R}}
\large

\textit{Extract the coefficients from the model then and exponentiate}
\begin{framed}
\begin{verbatim}
exp(coef(test))
 
          (Intercept) sesmiddle seshigh  write
 general        17.33    0.5867  0.3126 0.9437
 vocation      184.61    1.3383  0.3743 0.8926
\end{verbatim}
\end{framed}
\end{frame}
%================================================================================================%
\begin{frame}[fragile]
\frametitle{Multinomial Logistic Regression with \texttt{R}}
\Large
\begin{itemize}
\item The relative risk ratio for a one-unit increase in the variable \textbf{\textit{write}} is 0.9437 for being in general program vs. academic program. 
\item The relative risk ratio switching from \textbf{\textit{ses}} = 1 to 3 is 0.3126 for being in general program vs. academic program. 
\end{itemize}
\end{frame}
%================================================================================================%
\begin{frame}[fragile]

\frametitle{Multinomial Regression with \texttt{R}}
\Large
\begin{itemize}
\item You can also use predicted probabilities to help you understand the model. 
\item You can calculate predicted probabilities for each of our outcome levels using the \texttt{fitted} function. 
\item We can start by generating the predicted probabilities for the observations in our dataset and viewing the first few rows
\end{itemize}
\end{frame}
%================================================================================================%
\begin{frame}[fragile]

\frametitle{Multinomial Regression with \texttt{R}}
\Large
\begin{verbatim}
head(pp <- fitted(test))
 
   academic general vocation
 1   0.1483  0.3382   0.5135
 2   0.1202  0.1806   0.6992
 3   0.4187  0.2368   0.3445
 4   0.1727  0.3508   0.4765
 5   0.1001  0.1689   0.7309
 6   0.3534  0.2378   0.4088
\end{verbatim} 
\end{frame}
%================================================================================================%
\begin{frame}[fragile]
	
	\frametitle{Multinomial Regression with \texttt{R}}
	\Large
\textbf{Prediction}
\begin{itemize}
\item Suppose we want to examine the changes in predicted probability associated with one of our two variables, we can create small datasets varying one 
variable while holding the other constant. 
\item We will first do this holding \textbf{\textit{write}} at its mean and examining the predicted probabilities for each level of \textbf{\textit{ses}}.
\item \textit{(i.e. Three Cases to predict for) }
\end{itemize}

 
\end{frame}
%================================================================================================%
\begin{frame}[fragile]

\frametitle{Multinomial Logistic Regression with \texttt{R}}
\Large
\begin{verbatim}
dses <- data.frame(ses = c("low", "middle", "high"), 
              write = mean(ml$write))
predict(test, newdata = dses, "probs")
 
   academic general vocation
 1   0.4397  0.3582   0.2021
 2   0.4777  0.2283   0.2939
 3   0.7009  0.1785   0.1206
\end{verbatim}
\end{frame}
%================================================================================================%
\begin{frame}[fragile]

\frametitle{Multinomial Logistic Regression with \texttt{R}}
\Large
\begin{itemize}
\item Another way to understand the model using the predicted probabilities is to look at the averaged predicted probabilities for different values 
of the continuous predictor variable \textbf{\textit{write}} within each level of \textbf{\textit{ses}}.
\end{itemize}

 
\end{frame}
%================================================================================================%
\begin{frame}[fragile]

\frametitle{Multinomial Logistic Regression with \texttt{R}}
\large
Store the predicted probabilities for each value of ses and write
\begin{verbatim}
dwrite <- data.frame(
  ses = rep(c("low", "middle", "high"), each = 41),
  write = rep(c(30:70), 3))


pp.write <- cbind(dwrite, 
     predict(test, newdata = dwrite, 
     type = "probs", se = TRUE))
\end{verbatim}
\end{frame}
%================================================================================================%
\begin{frame}[fragile]

\frametitle{Multinomial Logistic Regression with \texttt{R}}
\large
 Calculate the mean probabilities within each level of ses
\begin{verbatim}
by(pp.write[, 3:5], pp.write$ses, colMeans)
 
 pp.write$ses: high
 academic  general vocation 
   0.6164   0.1808   0.2028 
 --------------------------------------
 pp.write$ses: low
 academic  general vocation 
   0.3973   0.3278   0.2749 
 --------------------------------------
 pp.write$ses: middle
 academic  general vocation 
   0.4256   0.2011   0.3733
\end{verbatim}
\end{frame}
%================================================================================================%
%\begin{frame}[fragile]
%
%\frametitle{Multinomial Regression with \texttt{R}}
%\Large
%\begin{itemize}
%\item A couple of plots can convey a good deal amount of information. 
%\item Using the predictions we generated for the \texttt{pp.write} object above, we can plot the predicted probabilities against the writing score by the level 
%of \textbf{\textit{ses}} for different levels of the outcome variable.
%\end{itemize} 
%\end{frame}
%================================================================================================%
%\begin{frame}[fragile]
%
%\frametitle{Multinomial Logistic Regression with \texttt{R}}
%\large
%remark: melt data set to long for ggplot2
%\begin{verbatim}
%lpp <- melt(pp.write, id.vars = c("ses", "write"),
%            value.name = "probability")
% 
%head(lpp)  # view first few rows
% 
%   ses write variable probability
% 1 low    30 academic     0.09844
% 2 low    31 academic     0.10717
% 3 low    32 academic     0.11650
% 4 low    33 academic     0.12646
% 5 low    34 academic     0.13705
% 6 low    35 academic     0.14828
%\end{verbatim}
%\end{frame}
%================================================================================================%
%\begin{frame}[fragile]
%
%\frametitle{Multinomial Regression with \texttt{R}}
%\Large 
% plot predicted probabilities across write values for each level of ses
% facetted by program type
%\begin{verbatim}
%ggplot(lpp, aes(x = write, y = probability, colour = ses)) + geom_line() + facet_grid(variable ~
%    ., scales = "free")
% 
%\end{verbatim}
%\end{frame}
\end{document}
